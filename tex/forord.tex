Velkommen til Den Frie Bibel.

Den Frie Bibel udspringer af ønsket om at skabe en tekstnær dansk bibeloversættelse som er fri for
enhver form for copyright. Altså en tekst som alle og enhver har lov til at kopiere og benytte som
de ønsker. Forbilledet har været den engelske World English Bible (https://ebible.org/web).

Foreløbig er det kun et begrænsket antal kapitler der foreligger i mere eller mindre færdiggjort
grad. Visse kapitler er oversat fra grundsproget, andre er en moderniseret og revideret udgave af de
danske oversættelser fra 1871 (GT) og 1907 (NT).

Oversættelsen er tekstnær. Det betyder at der er lagt vægt på at den danske tekst skal være en så
præcis gengivelse af den hebraiske tekst som muligt, også på steder hvor den hebraiske tekst er
vanskelig at forstå. Hvor originalen er tvetydig, bør oversættelsen også være det. Der er altså ikke
tale om en gendigtning af den bibelske tekst.

Oversættelsen vil løbende blive revideret og opdateret, det er derfor vigtigt at være opmærksom på
den dato der står på forsiden af denne udgave.

Kommentarer og spørgsmål til projektet kan sendes til Claus Tøndering (claus@tondering.dk).
